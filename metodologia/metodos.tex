\mychapter{Metodologia}
\label{cap_3}

\section{Estudo e desenvolvimento inicial}


\subsection{Recursos e estrategias utilizadas no desenvolvimento}

	
O sistema  Windows foi adotado como sistema operacional para o desenvolvimento do aplicativo, por ser um sistema bastante difundido e familiar, no entanto, nada impede que haja portabilidade dos aplicativos para outros sistemas operacionais, pois foi utilizado a plataforma Java. A IDE de desenvolvimento do software foi o NetBeans 8.2 da  Oracle. A linguagem de programação Java foi a escolhida para as  dos programas desenvolvidos, devido

%A linguagem de programação Java foi a escolhida para as implementações dos programas desenvolvidos, devido à facilidade na orientação a objeto, o que possibilitou a replicação de blocos e possíveis integrações com o Matlab caso fosse necessário..
%A estratégia utilizada na implementação dos algoritmos distingue a abordagem segundo a Inteligência Artificial Simbólica (implementações de conversões simbólicas baseadas na Lógica) e a Conexionista (implementações enfocavam as RNAs). Como último ponto da estratégia, houve a integração dos dois paradigmas, através das implementações dos algoritmos Neuro-Simbólicos do KBANN.
%O método utilizado para a implementação foi baseado em testes, primeiramente focando problemas específicos, para, logo em seguida, desenvolver programas que satisfaçam os requisitos da classe desses problemas. O último passo foi a ampliação do programa para o atendimento de outras classes de problemas, tornando-o mais genérico.



