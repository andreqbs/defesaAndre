%
% ********** Resumo
%

% Usa-se \chapter*, e n�o \chapter, porque este "cap�tulo" n�o deve
% ser numerado.
% Na maioria das vezes, ao inv�s dos comandos LaTeX \chapter e \chapter*,
% deve-se usar as nossas vers�es definidas no arquivo comandos.tex,
% \mychapter e \mychapterast. Isto porque os comandos LaTeX t�m um erro
% que faz com que eles sempre coloquem o n�mero da p�gina no rodap� na
% primeira p�gina do cap�tulo, mesmo que o estilo que estejamos usando
% para numera��o seja outro.
\mychapterast{Resumo}
Sistemas inteligentes (SI) s�o tipos de sistemas especialistas que podem interagir e aprender sobre os usu�rios que o utilizam, ou tentam compreender sobre seus interesses. O termo SI engloba tanto a intelig�ncia artificial cl�ssica, baseado em regras simb�licas, quanto a intelig�ncia computacional, que incorpora sistemas nebulosos, redes neurais artificiais (RNA) e sistemas evolutivos. Os sistemas inteligentes simb�licos possuem um conjunto de regras inicias que s�o continuamente analisadas, gerando decis�es a partir delas. Embora mais simples de se compreender, esses sistemas possuem dificuldades em tratar informa��es imprecisas, ou que n�o foram previstas no conjunto de regras iniciais. Por outro lado, os modelos baseados no conexionismo, como por exemplo, as RNAs, s�o extremamente eficazes em completar padr�es que n�o estejam claros, embora, possuam uma grande desvantagem devido ao elevado n�vel de abstra��o, encapsulando o conhecimento de como foi obtida a resposta. Esses dois tipos de sistemas podem ser combinados para suprir as desvantagens apresentadas em cada um deles, produzindo assim uma poderosa ferramenta no �mbito industrial. O presente trabalho, tem por finalidade descrever uma arquitetura de um sistema computacional h�brido, combinando os principais paradigmas da intelig�ncia artificial (IA): o simbolismo e, o conexionismo. O modelo foi capaz de codificar a base regras de um especialista (baseado em l�gica proposicional) em uma rede neural e a partir dela inferir novas regras de acordo com a utiliza��o do sistema, servindo assim de aux�lio para as constantes tomadas de decis�o do operador.
%Conclusao (apresentar resultados)
 \vspace{1.5ex}


{\bf Palavras-chave}: Neuro-simb�lico, sistema inteligente h�brido, representa��o de conhecimento h�brido.
%
% ********** Abstract
%

\mychapterast{Abstract}
One of the main goals of artificial intelligence is the creation of agents with human-like intelligence. This has been researched using various approaches, and among the most prominent  for machine learning are logic-based symbolic systems and artificial neural networks. Until the last decade, both approaches have progressed independently, but progress in both areas has led researchers to investigate ways to integrate both approaches. Several models that provide hybrid or integrated integration of these approaches emerged in the 1990s, and continue to be used to this day. This work has as main objective the implementation of the Neuro-Symbolic conversion algorithm of the KBANN (Knowledge-Based Artificial Neural Networks) hybrid system, which has the ability to map the dependencies of a specific domain of rules (if-then) in a Neural network, and then refine that network using learning techniques. In addition, since this model does not have the ability to refine the network topology in order to obtain new rules for the initial domain, it was necessary to implement another algorithm to the original KBANN model, in order to obtain possible expansions of the original network. KBANN.
\vspace{1.5ex}

{\bf Keywords}: Linear Systems Subject to Constraints, Multi-parametric Programming, Cluster Analysis, Positively Invariant Sets.
