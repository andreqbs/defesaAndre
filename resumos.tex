\mychapterast{Resumo}
Um dos principais objetivos da intelig�ncia artificial � a cria��o de agentes com a intelig�ncia humana. Isso vem sendo pesquisado utilizando v�rias abordagens, e entre as mais promissoras para o aprendizado de m�quinas s�o os sistemas simb�licos baseados na l�gica e as redes neurais artificiais. At� a �ltima d�cada, ambas as abordagens progrediam de forma independente, mas os progressos obtidos em ambas as �reas fizeram com que os pesquisadores come�assem a investigar maneiras de integrar as duas t�cnicas. Diversos modelos que proporcionam a integra��o h�brida ou integrada desses m�todos inteligentes surgiram na d�cada de 90, e continuam sendo utilizadas e melhoradas at� hoje. 

Esse trabalho tem como objetivo principal a implementa��o e uso do algoritmo de convers�o Neuro-Simb�lica do sistema h�brido KBANN (\emph{Knowledge-Based Artificial Neural Networks}), o sistema possui a capacidade de mapear um dom�nio te�rico espec�fico de regras (se-ent�o) em uma rede neural, e refinar a rede utilizando t�cnicas de aprendizado. Al�m disso, como o algoritmo criado por \cite{towell90} n�o possui a capacidade de adquirir novos conhecimentos e introduzi-los a rede neural, ser� utilizado o algoritmo TopGen \cite{shavlik95} para adicionar tal capacidade a rede sem perder o conhecimento original adquirido. O trabalho utilizou um jogo de tabuleiro para realizar experimentos devido as regras bem estabelecidas do jogo. O sistema implementado conseguiu resultados interessantes, mesmo com a pertuba��o do dom�nio inicial de regras (com a exclus�o delas), obtendo uma taxa de acerto pr�xima a 100\%. Portanto, a partir dos resultados obtidos foi poss�vel concluir que os sistemas h�bridos s�o capazes de se sobrepor a situa��es adversas as quais foram realizadas as an�lises propostas nessa pesquisa.
 \vspace{1.5ex}

{\bf Palavras-chave}: Neuro-simb�lico, sistemas h�brido, representa��o de conhecimento em uma rede neural.

\mychapterast{Abstract}
One of the main goals of artificial intelligence is the creation of agents with human-like intelligence. This has been researched using various approaches, and among the most prominent  for machine learning are logic-based symbolic systems and artificial neural networks. Until the last decade, both approaches have progressed independently, but progress in both areas has led researchers to investigate ways to integrate both approaches. Several models that provide hybrid or integrated integration of these approaches emerged in the 1990s, and continue to be used to this day. This work has as main objective the implementation and use of the Neuro-Symbolic conversion algorithm of the KBANN (Knowledge-Based Artificial Neural Networks) hybrid system, which has the ability to map a specific theoretical domain of rules (if-then) in a Neural network, and refine the network using learning techniques. In addition, because the algorithm created by \cite{towell90} does not have the ability to acquire new knowledge and introduce the neural network, the algorithm TopGen \cite{shavlik95} will be used to add such capacity to the network without losing the original knowledge acquired. The paper used a board game to conduct experiments due to the well established rules of the game. The implemented system achieved interesting results, even with the initial rule domain perturbation (with the exclusion of them), obtaining a hit rate close to 100 \%. Therefore, from the obtained results it was possible to conclude that the hybrid systems are able to overlap to adverse situations that the analyzes proposed in this research were carried out.
\vspace{1.5ex}

{\bf Keywords}: Neural-symbolic, hybrid systems, representation of knowledge in a neural network.
